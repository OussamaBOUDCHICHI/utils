\documentclass[a4paper,12pt, openright]{report}

% Packages francais
%\usepackage[french]{babel}
\usepackage[T1]{fontenc}
\usepackage[utf8]{inputenc}
\usepackage{lmodern}
\usepackage{bbm}
\usepackage{stmaryrd}
\usepackage[displaymath, mathlines]{lineno}
\usepackage{minitoc}
%\usepackage{bbold}
%\usepackage{amssymb}
\usepackage{amsmath, amsfonts, dsfont, amsthm}
\usepackage{mathtools}
\usepackage{stackrel}
\usepackage[ruled,vlined]{algorithm2e}
\usepackage[outline]{contour}

%\setlength\arrayrulewidth{1.1pt}


% Document starts here
% 
\newcounter{daggerfootnote}
\newcommand*{\daggerfootnote}[1]{%
    \setcounter{daggerfootnote}{\value{footnote}}%
    \renewcommand*{\thefootnote}{\fnsymbol{footnote}}%
    \footnote[2]{#1}%
    \setcounter{footnote}{\value{daggerfootnote}}%
    \renewcommand*{\thefootnote}{\arabic{footnote}}%
    }
    
\usepackage[ruled,vlined, french]{algorithm2e}
\usepackage{titlesec}
\usepackage[top=1cm, bottom=1cm, margin=1.8cm]{geometry}
\newenvironment{allintypewriter}{\ttfamily}{\par}
\usepackage{stmaryrd}
\usepackage{caption}
\usepackage{xcolor}
\usepackage{graphicx}
% Personnalisation des marges et haut/bas de pages
\usepackage{fancyhdr}
%\usepackage[left=2cm,right=2cm,top=2cm,bottom=2cm]{geometry}

% Packages dessins
\usepackage{pict2e}
\usepackage{tikz}
\usepackage{graphicx}

% Packages Ams
\usepackage{amsthm}
\usepackage{amssymb}
\usepackage{amsmath}
\usepackage[Sonny]{fncychap}
\ChNameVar{\Large\sc} \ChNumVar{\Huge} \ChTitleVar{\Huge\bf} \ChRuleWidth{1pt} \ChNameUpperCase
\makeatletter
\def\vhrulefill#1{\leavevmode\leaders\hrule\@height#1\hfill \kern\z@}
\makeatother
\setlength\arrayrulewidth{1.1pt}

% Package pour les hyperliens
\definecolor{bv}{rgb}{0.54, 0.17, 0.89}
\definecolor{bz}{rgb}{0.74, 0.2, 0.64}
\definecolor{indigo}{rgb}{0.29, 0.0, 0.51}
\definecolor{persIn}{rgb}{0.2, 0.07, 0.48}
\definecolor{purple}{rgb}{0.5, 0.0, 0.5}
\usepackage[colorlinks=true, urlcolor=blue, linkcolor =myred, citecolor=purple, ]{hyperref}
\usepackage{amsfonts}
% Package pour les intervalles d'entiers
\usepackage{stmaryrd}

%Package pour enumerate
\usepackage{enumitem}

% Numerotation des equations
\numberwithin{equation}{section}

% Styles de texte
\newcommand \dis {\displaystyle}
\newcommand \tex {\textstyle}
\newcommand \scr {\scriptstyle}

% Flèches convergences: suites, familles
\newcommand \limo {\;\;\mathop{\longrightarrow}_{\e\to 0}\;\;}
\newcommand \limod {\;\;\mathop{\longrightarrow}_{\de\to 0}\;\;}
\newcommand \limh {\;\;\mathop{\longrightarrow}_{h\to 0}\;\;}
\newcommand \limi {\;\;\mathop{\longrightarrow}_{n\to+\infty}\;\;}
\newcommand \limk {\;\;\mathop{\longrightarrow}_{k\to+\infty}\;\;}
\numberwithin{equation}{section}

%% Colors ====================
\definecolor{darkWhite}{rgb}{0.94,0.94,0.94}
\definecolor{dg}{rgb}{0.0, 0.5, 0.0}
\definecolor{dr}{rgb}{0.7, 0.11, 0.11}
\definecolor{db}{rgb}{0.0, 0.0, 0.55}
\definecolor{mb}{rgb}{0.0, 0.0, 0.8}
\colorlet{myred}{red!60!black}
\definecolor{sb}{rgb}{0.14, 0.16, 0.48}
\definecolor{NB}{rgb}{0.0, 0.0, 0.5}
\definecolor{midnightblue}{rgb}{0.1, 0.1, 0.44}
\definecolor{specCol}{HTML}{B42C4E}
\definecolor{myblue}{HTML}{28629D}
\definecolor{pc}{HTML}{800040}
\definecolor{byz}{rgb}{0.44, 0.16, 0.39}
% ===========================

% Norme triple
\newcommand\tn{|\!|\!|} 
\DeclareMathOperator*{\argmax}{arg\,max\,}
\DeclareMathOperator*{\argmin}{arg\,min\,}
% Lettres grecques
\newcommand \al {\alpha}
\newcommand \be {\beta}
\newcommand \ga {\gamma}
\newcommand \Ga {\Gamma}
\newcommand \de {\delta}
\newcommand \De {\Delta}
\newcommand \vare {\varepsilon}
\newcommand \om {\omega}
\newcommand \Om {\Omega}
\newcommand \la {\lambda}
\newcommand \La {\Lambda}
\newcommand \ph {\varphi}
\renewcommand \th {\theta}
\newcommand \ka {\kappa}
\newcommand \si {\sigma}
\newcommand \SSi {\Sigma}
\newcommand \grad {\nabla}

% Ensembles classiques en gras
\newcommand*\dd{\mathop{}\!\mathrm{d}}
\newcommand*\R{\mathop{}\!\mathbb{R}}
\newcommand*\N{\mathop{}\!\mathbb{N}}

\newcommand*\E{\mathop{}\!\mathbb{E}}
\newcommand*\F{\mathop{}\!\mathbb{F}}
\newcommand*\PP{\mathop{}\!\mathbb{P}}
\newcommand*\U{\mathop{}\!\mathcal{U}}
\newcommand*\OO{\mathop{}\!\mathcal{O}}
\newcommand*\HH{\mathop{}\!\mathbb{H}}
\newcommand{\1}{\mathbbm{1}}
\newcommand{\indep}{\perp \!\!\! \perp}
% Ensembles/familles avec ecritures rondes
\renewcommand \AA {\mathcal{A}} % La commande \AA existe deja de base
\newcommand \BB {\mathcal{B}}
\newcommand \DD {\mathcal{D}}
\newcommand \EE {\mathcal{E}}
\newcommand \FF {\mathcal{F}}
\newcommand \GG {\mathcal{G}}
\newcommand \II {\mathcal{I}}
\newcommand \JJ {\mathcal{J}}
\newcommand \KK {\mathcal{K}}
\newcommand \LL {\mathcal{L}}  % La commande \L existe deja de base
\newcommand \GL {\mathcal{G}\mathcal{L}} % Groupe libre
\newcommand \MM {\mathcal{M}}
\newcommand \RR {\mathcal{R}}  
\renewcommand \SS {\mathcal{S}} 
\newcommand \TT {\mathcal{T}}
\newcommand \UU {\mathcal{U}}
\newcommand \VV {\mathcal{V}}

% Quelques raccourcis utiles
\newcommand \beq {\begin{equation}} % environnement equation
\newcommand \eeq {\end{equation}}
\newcommand \ba {\begin{array}} % environnement tableau
\newcommand \ea {\end{array}}
\newcommand \ecart {\noalign{\medskip}}% grand saut de ligne
\newcommand{\diag}{\text{diag}} % matrice diagonale

%\frenchbsetup{StandardItemLabels=true}

\newenvironment{questions}{\begin{enumerate}[label={\bf\arabic*)}]}{\end{enumerate}}
\newenvironment{subquestions}{\begin{enumerate}[label={\bf\alph*)}]}{\end{enumerate}}


\AtBeginDocument{\def\labelitemi{$\rhd$}}

% Définitions des environnements Théorème, Corollaire, Définition, etc.

\newtheoremstyle{mystyle}%                % Name
  {}%                                     % Space above
  {}%                                     % Space below
  {\normalfont}%                                     % Body font
  {}%                                     % Indent amount
  {\color{pc}\scshape}%                            % Theorem head font
  {.}%                                    % Punctuation after theorem head
  { }%                                    % Space after theorem head, ' ', or \newline
  {\thmname{#1}\thmnumber{ #2}\thmnote{ (#3)}}%                                     % Theorem head spec (can be left empty, meaning `normal')

\theoremstyle{mystyle}
\newtheorem{theorem}{Theorem}[subsection]
\newtheorem{theo}{Theorem}[section]
\newtheorem{prop}{Proposition}[subsection]
\newtheorem{definition}{Definition}[subsection]
\newtheorem{lemme}{Lemma}[subsection]
\newtheorem{defi}{Definition}[section]
\newtheorem{propo}{Proposition}[section]
\newtheorem{cor}{Corollary}[subsection]


%\newtheorem{Pres}[Theoreme]{Preuves}
%\newtheorem{proof}[Theoreme]{Preuve} 
 


\usepackage{blindtext}
\newcommand{\prooffont}{\scshape}


\usepackage{xpatch}
\tracingpatches
\xpatchcmd{\proof}{\itshape}{\prooffont}{}{}

% Raccourcis pour les references}
\newcommand \refe[1]{(\ref{#1})}
\newcommand \reff[1]{figure~\ref{#1}}
\newcommand \refs[1]{Section~\ref{#1}}
\newcommand \refss[1]{subsection~\ref{#1}}
\newcommand \refD[1]{Définition~\ref{#1}}
\newcommand \refT[1]{Théorème~\ref{#1}}
\newcommand \refL[1]{Lemme~\ref{#1}}
\newcommand \refC[1]{Corollaire~\ref{#1}}
\newcommand \refP[1]{Proposition~\ref{#1}}
\newcommand \refPt[1]{Propriétés~\ref{#1}}
\newcommand \refR[1]{Remarque~\ref{#1}}
\newcommand \refE[1]{Exemple~\ref{#1}}
\newcommand \refN[1]{Notation~\ref{#1}}


% Raccourcis pour ecrire u et f en gras
\newcommand{ \uu} {\boldsymbol{u}}
\newcommand{ \ff} {\boldsymbol{f}}
\newcommand{\VaR} {\text{VaR}}
\newcommand{\CVaR} {\text{CVaR}}

% Raccourcis vectoriel
\newcommand \Vect[1]{\text{Vect}(#1)}
\newcommand \Ker[1]{\text{Ker}(#1)}
\newcommand \im[1]{\text{Im}(#1)}
\newcommand\fIf[2]{\left\llbracket #1;#2\right\rrbracket}

% Le dx droit dans l'intégrale
\newcommand \dx {\mathrm{d}x}

% L'exponentiel 'droit'
\DeclareMathOperator{\e}{e}
\DeclareMathOperator{\supess}{supess}

% Titre, auteur et date sont utilises plus bas dans le titlepage
\title{\textsc{Path-Dependent Local Volatility models : theoretical study and backtests} }
\author{ \textsc{Oussama BOUDCHICHI}}
\date{2022-2023}
%\newcommand{\chaptertoc}[1]{\chapter*{#1}% creation d'une commande chaptertoc pour ne pas numeroter le chapitre intro

% Style de pages: definit les hauts et bas (ici vide) de pages
\pagestyle{empty}
\pagestyle{fancyplain}
%\renewcommand{\headrulewidth}{1pt}
\renewcommand{\footrulewidth}{0pt}
\fancyhf{}
%\fancyhead[LE,RO]{\thepage}  %LE=Left Even, RO=Right Odd
\fancyhead[LO]{{\footnotesize\textsc{\rightmark}}}
\fancyhead[RE]{{\footnotesize\textsc{\leftmark}}}
\cfoot{\thepage}
%\addtolength{\headheight}{25pt}
\renewcommand\qedsymbol{$\textcolor{myred}{\square}$}
\usepackage[toc,page]{appendix}

%%%%%%%%%%%%%%%%%%%%%%%%%%%%%%%%%%%%%%%%%%%%%%%%%%%%%%%%%%
%% Document starts here 
%%%%%%%%%%%%%%%%%%%%%%%%%%%%%%%%%%%%%%%%%%%%%%%%%%%%%%%%%%
%\renewcommand{\thesection}{\arabic{section}}
\usepackage{titlesec}
\titleformat*{\section}{\Large\scshape}
\titleformat*{\subsection}{\scshape\color{db}}
\titleformat*{\subsubsection}{\scshape\color{specCol}}

\makeatletter
\newcommand\xparagraph{\@startsection{paragraph}{4}{\z@}%
                                    {3.25ex \@plus1ex \@minus.2ex}%
                                    {-1em}%
                                    {\scshape}}
\makeatother
\newtagform{red}{\color{myred}(}{)}



\begin{document}
%\maketitle
\usetagform{red}

\makeatletter

  \begin{titlepage}
\newgeometry{left=0.25cm,right=0.25cm,top=0.25cm, bottom=0cm}
    %\vspace{-5cm}
      %
	\centering
	\includegraphics[width=0.4\textwidth]{UniversiteParisCite_logo_horizontal_couleur_CMJN}
  \hfill
  
	\par\vspace{3cm}
	{\scshape\Large Master's Thesis \par}
	
	\vspace{2cm}
            %\hrule height 2pt 
            \hspace{0.02\paperwidth}\textcolor{pc}{\vhrulefill{2pt}}\hspace{0.02\paperwidth}

            \vspace*{0.6cm}
		{\huge \@title \par}
            \vspace*{0.6cm} \hspace{0.02\paperwidth}\textcolor{pc}{\vhrulefill{2pt}}\hspace{0.02\paperwidth}
	\vspace{1cm}
        
	{\LARGE \@author \par}
 
	\vspace{2cm}
	{\large Supervised by \par
        \textsc{Alexey KOZHEMYAK}\par
	\textsc{Jean-François CHASSAGNEUX}\par}
	\vspace{2cm}
	{\large \@date\par}
	\vspace{2cm}
	{\scshape\Large Université Paris Cité \par}
	\vfill
\end{titlepage}  
\makeatother
\restoregeometry

\pagestyle{fancy}
\fancyhf{}
\cfoot{\thepage}

${}^{}$


\renewcommand{\abstractname}{Abstract}

\begin{abstract}
    \pagestyle{fancy}
\fancyhf{}
\cfoot{\thepage}

   
\end{abstract}
 %Table des matières
\newpage
\pagestyle{fancy}
\fancyhf{}
\cfoot{\thepage}
\setcounter{tocdepth}{1}
\begingroup
\hypersetup{linkcolor=midnightblue}
\tableofcontents
\endgroup

\begingroup
\hypersetup{linkcolor=midnightblue}
\listoffigures
\endgroup

\iffalse
\chapter*{List of notations}
\pagestyle{fancy}
\fancyhf{}
\cfoot{\thepage}

%\addstarredchapter{Liste des notations} 

\begin{enumerate}
\item[•] For $(x, y) \in \R^2$, $x\wedge y := \min\left(x, y\right)$ and $x\vee y := \max\left(x, y\right)$\\

\iffalse
\item[•] $\mathbb{H}^p\left(\mathbb{F}\right)$ : L'ensemble des processus $\nu$, $\mathbb{F}$-progressifs tel que : $\mathbb{E}\left(\displaystyle \int_{0}^{T}\left| \nu_t\right|^p \dd t \right) < \infty$ pour $p>0$.\\
\fi
\item[•] $| \sigma | := \left( \mathrm{Tr}\left(\sigma \sigma^T\right)\right)^{\frac{1}{2}}$ is the Frobenius norm of a square matrix $\sigma$.\\

\item[•] $\langle x, y \rangle := \displaystyle \sum_{j = 1}^{d}x_jy_j$ for $x, y \in \R^d$. We denote $|.|$ its associated norm.\\

\item[•] The interior, the closure and the boundary of a set  $\mathcal{O}$ are denoted  respectively $\overset{o}{\mathcal{O}}$, $\overline{\OO}$ and $\displaystyle \partial\OO$.\\


\item[•] $\mathcal{S}_d$ is the set of square real valued symmetric matrices of order $d$.\\

\item[•] $\nabla_x \psi(t, x) = \left(\partial_{x_j}\psi(t,x)\right)_{j\in \llbracket 1,d\rrbracket}$ and $H_x \psi(t, x) = \left(\partial_{x_ix_j}^2 \psi(t, x) \right)_{(i, j) \in \llbracket 1,d\rrbracket^2}$ for $(t, x) \in \R_+ \times \R^d$ are respectively the gradient and the hessian matrix w.r.t the space variable of a function $\psi \in \mathcal{C}^{1,2}\left(\R_+^* \times \R^d\right)$. \\

\item[•] $\left\{f = 0\right\} := \left\{ x\in \R^d: \quad f(x) = 0_{\R^n}\right\}$ such that $f: \R^d \to \R^n$. 
\end{enumerate}
\newpage
\fi

\chapter*{Introduction}
\addstarredchapter{Introduction}
\pagestyle{fancy}
\fancyhf{}
\cfoot{\thepage}


\newpage

\iffalse
\pagestyle{fancy}
\fancyhf{}
\cfoot{\thepage}

\vspace*{0.3\textheight}
\begin{center}
    \includegraphics[scale = 3]{Map.pdf}
\end{center}
\fi
\newpage
%%%%%%%%%%%%%%%%%%%%%%%%%%%%%%%%%%%%%%%%%%%%
%%%%%%%%%%%%%%%%%%%%%%%%%%%%%%%%%%%%%%%%%%%%

%%%%%%%%%%%%%%%%%%%%%%%%%%%%%%%%%%%%%%%%%%%%
%%%%%%%%%%%%%%%%%%%%%%%%%%%%%%%%%%%%%%%%%%%%
\chapter{Path-Dependent Volatility modelling in a nutshell}
\pagestyle{fancyplain}
\renewcommand{\headrulewidth}{1pt}
\renewcommand{\footrulewidth}{1pt}
\fancyhf{}
%\fancyhead[LE,RO]{\thepage}  %LE=Left Even, RO=Right Odd

\fancyhead[L]{\fontsize{9}{12} \selectfont \rightmark}
%\fancyhead[R]{\thepage}
\cfoot{\thepage}
\renewcommand{\subsectionmark}[1]{%
  \markright{\MakeUppercase{\thesubsection.\ #1}}}%

%\fancyhead[LO]{{\footnotesize\textsl{\rightmark}}}
%\fancyhead[RE]{{\footnotesize\textsl{\leftmark}}}
%\cfoot{\thepage}







\appendix
\chapter{Proof}\label{appendix:sa}

%%%%%%%%%%%%%%%%%%%%%%%%%%%%%%%%%%%%%%%%%%%%%%%%%%%%%
%% Bibliography %%%%%%%%%%%%%%%%%%%%%%%%%%%%%%%%%%%%%

\bibliographystyle{apalike}
\addstarredchapter{Bibliography} 
\bibliography{refs}   
\pagestyle{fancy}
\fancyhf{}
\cfoot{\thepage}
\end{document}

